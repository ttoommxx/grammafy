\begin{document}
some gibberish text \ciaocome \ciaocomestai{s}[]

[\textbf{ciao come stai}]
[\textbb{bene e tu?}]
\unknownCommand{come stai {io}}

Let see if this also works \notAcommand{hey P{maybe} cazzo}

hey \mbox{ hello{A} := }.


\{ e questo funzione? \} maybe \"hello


ora aggiungiamo qualcos'altro \nonSo\textbf{hey}

come \begin{gather*}
    SISISISISISI    
    NONONONONONONO
\end{gather*}yes

\textbf$i don't know$

\hello{prova1} ee[]
\hello{prova2}{ciao}
\hello*{come}[ee]{stay}ciao

questo dovrebbe funzionare

\begin{align*}
e^i+1=0 hello dear
\end{align*}
and this is the end

come stau?

prima___11111
\begin{align}
Io andrei al passeggio ahahah,
\end{align}
dopo____22222
e domani mangiamo la\textbf{pastaasciutto}

ecco altro testo que {qui}


\begin{equation} hello \end{equation}
\begin{equation*} hello \end{equation*}

again! \centering

the beginning of the theorem {\Large here}
\begin{theorem}
I want to show that $1+1-1 = 1$
\end{theorem}
\begin{proof}
because $1-1=0$ then $1+(1-1)=1+0=0.$
\end{proof}

\section*{now we do something else}

ciao! Now let's try {\small two different equations}

first with the double dollar
$$
this is it!,
$$
did it work?

\[ hello, \] is this now working?? \\ want to write something in red {\color{red} MOLTO ROSSO }

\[
there was a strange problem in the middle.    
\]

\begin{itemize}

\item pasta $v \in here does it now work???$
\item asciutta

\item carrots
\end{itemize} he

-------------------------------------------------------------

