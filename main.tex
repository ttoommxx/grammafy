\section{Intro}

\subsection{Terminology and symbols}

\begin{itemize}
\item For a vector $v\in \R^m$ we write $|v| = \sqrt{\sum_{i=1}^m v_i^2}$, otherwise specified.
\item The Lebesgue measure will be indicated by $dx$ or $\L^n$ depending on the context. The set of finite Borel m-vector measures is indicated by $\M(\Omega,\R^d)$. For $E\subset \M(\Omega)$, $E^+$ is the set of positive Borel measures that belong to $E$.
\item For a measure $\mu\in \M(\Omega)^+$ we write $\L^p(\Omega,\mu,\R^d)$ to mean the space of $\mu$-measurable functions $f \colon \Omega\to \R^m$ such that
\begin{align}
\int_\Omega |f(x)|^pd\mu(x) <+ \infty.
\end{align}
\item For $\mu\in \M(\Omega)$ we write its restriction to $E\subset \Omega$ Borel set by
\begin{align}
\mu\mres E \colon A \text{ Borel set} \mapsto \mu(E\cap A).
\end{align}
\item If $X$ is any space, the Dirac delta is indicated, for $x\in X$,
\begin{align}
\int_X f(y) d\delta_x(y) = f(x),
\end{align}
for all functions $f\colon X\to \R^d$.
\item \begin{align}
\D(\Omega,\R^d) = C^\infty_c(\Omega,\R^d) = \{ f:\Omega\to \R^d: f \text{ is infinitely differentiable and has compact support} \}.
\end{align}
It's dual $\D(\Omega,\R^d)^*$ is the set of distributions.
\item Throughout this thesis, for the derivative of a function $u\in L^1(\Omega,\R^m)$ we mean a matrix valued distribution $Du = [\partial_j u_i]_{i,j} \in (\D(\Omega,\R^m)^*)^n$ such that
\begin{align}
\int_\Omega u_i \partial_j \phi dx = - \langle \partial_j u_i,\phi\rangle.
\end{align}
\item \begin{align}
W^{1,1}(\Omega,\R^m) = \{ u\in L^1(\Omega,\R^m) : Du\in L^1(\Omega,\R^{m\times n}) \}.
\end{align}
\item \begin{align}
BV(\Omega,\R^m) = \{ u\in L^1(\Omega,\R^m) : Du\in \M(\Omega,\R^{m\times n}) \}.
\end{align}
\item For a function $u\in BV(\Omega,\R^m)$ we can write
\begin{align}
Du = \nabla u d\L^n\mres \Omega + D^s u, \quad D^s u = D^j u\mres J_u + D^c u
\end{align}
where $\nabla u d\L^n$ is the absolutely continuous part, $D^j u$ is the jump part and it is concentrated on a $n-1$ rectifiable set $J_u$ and $D^s$ is the Cantor part, and it is absolutely continuous with respect to $\H^{n-1}$.
\item For a function $U\in BV(\Omega,\R^m)$ we let
\begin{align}
BV_U(\Omega,\R^m) = & \{ u\in BV(\Omega,\R^m): \text{there is a sequence } u_j\in \D(\Omega,\R^m) \\
& \quad \text{ such that } u_j \xrightarrow{\text{weak* in BV}} u-U  \},
\end{align}
the closure being taken with respect to the weak* topology in $BV$.
\item \begin{align}
SBV(\Omega,\R^m) = \{ u\in BV(\Omega,\R^m): D^s u = D^j u \text{ or equivalently } D^c u=0\}.
\end{align}
\item We say that a function $f:\R^d\to \R$ has $p$ growth if there is $C>0$ such that
\begin{align}
|f(z)| \leq C(1+|z|^p).
\end{align}
\item The identity matrix is indicated by $\1$, or $\1_d$ to specify that $\1_d\in \R^{d\times d}$.
\item YM stands for Young Measure and GYM for grandient Young Measures, those Young Measures that are generated by sequences of gradients.
\end{itemize}

\newpage

\subsection{Introduction}
%
%In the following report, we introduce a generalise
%
%we study lower semi-continuous extensions of functionals of the form
%\begin{equation}
%u\mapsto \int_\Omega f(x,u(x),\frac{Du}{\mu}(x))d\mu(x),
%\end{equation}
%where $\mu\in \M^+(\Omega)$ is a suitable measure, $u$ smooth or in some Sobolev space depending on $\mu$ and the extension is carried to $BV(\Omega,\R^m)$. The non-linearity of the integrand $f$ makes it harder to study this problem, as concentration phenomena give rise to non-linear interactions between $u$, $Du$ and $f$ itself. For this reason, we introduce a new generalisation of Young Measure that gives an insight on the structure of limits of
%\begin{equation}
%f(x,u_j(x),\frac{Du_j}{\mu}(x))
%\end{equation}
%where $u_j$ is let converge in some weak sense to a function $u\in BV(\Omega,\R^m)$.

Young Measures were first introduced by Young in \cite{young1937generalized} to study minima of integral energies of the form
\begin{equation}
\inf\left\{ \int_0^1 f(u(t),u'(t))dt: u\in C^1([0,1]), u(0)=a, u(1)=b, \|u'\|_\infty \leq K \right\}.
\end{equation}
The author wanted to understand what conditions on $f$ would guarantee the existence of a minimizing curve $u(t)$. Young had the brilliant intuition that, for an extremely general class of functions $f$, minimising sequences always converge to a "generalised" curve $t\mapsto (u(t),\nu_t)$ where $\nu$ is a probability measure on the image of $f$. This translates to
\begin{equation}
\inf_{u(0)=1,u(1)=b} \int_0^1 f(u(t),u'(t))dt = \lim_j \int_0^1 f(u_j(t),u_j'(t))dt = \int_0^1 \int_\R f(u(t),y)d\nu_t(y) dt.
\end{equation}
So the question of the existence of a minimiser can be reformulated as to whether such objects are gradients of a curve or not. $\nu_t$ might fail to be a gradient when the minimizing sequence oscillates and so dissipates on the target space.

{\color{red} mention the lecture notes by Jan + di perna majda just mentioned there}

His original work focused on the case $n=1$ and was carried out via functional analytic methods. This approach was later extended in \cite{ball1989version,berliocchi1973integrandes} to higher dimensions. We call these generalised functions "oscillation Young Measures". The method developed by Young is not powerful enough to tackle problems arising in modern mathematics, as it can only handle sequences $(v_j)_{j\in\N}$ that are only bounded in $L^\infty$ rather than in some Lebesgue space. The first attempt to well represent generalised limits of integrable functions is due to DiPerna and Majda, \cite{diperna1987oscillations}. The idea of the authors was not too far from Young's work. Functions $v_j\colon \Omega\to \R^d$ are seen as Dirac deltas on the product space $\Omega\times \R^d$, which is subsequently compactified. An accumulation point, in the sense of these new generalised functions, is then found. Such accumulation point is a measure on an abstract compactification of $\Omega\times \R^d$, and as such, it is not clear how to represent it on the original, non-compact, space. In \cite{alibert1997non}, an explicit formula for such accumulation points was obtained for a class of integrands that have some sort of nice asymptotic when they grow at infinity.

In what follows, we give a general formula for limit Young Measures for a large class of integrands. This generalisation is based on the canonical way of constructing Hausdorff compactifications starting from continuous functions, see \cite{chandler1976hausdorff}. A small reduction lemma gives a clearer, more geometrical way of representation of these limits when the class of integrands is not too large. This formula captures oscillations at infinity, which are now let occur. We also prove few structure theorems that related different compactifications and Young Measures representations to each other. \\

The main reason why we construct such generalisation is to study variations and minima - within the class of functions of bounded variations $BV(\Omega,\R^m)$ - of energies that depend on gradients
\begin{equation} \label{Introduction:eq:functional_gradient}
u\mapsto \int_\Omega f(Du(x))dx,
\end{equation}
where $u\in \D(\Omega)$ and $f$ has suitable growth assumptions so that the integral makes sense. To better understand the existence of minima points, one is inclined to study whether \eqrefp{Introduction:eq:functional_gradient} is lower semi-continuous. For this type of functionals, convexity of $f$ is sufficient but not necessary for lower semi-continuity with respect to "relevant" topologies. In \cite{morrey1952}, Morrey established the equivalence of lower semi-continuity of \eqrefp{Introduction:eq:functional_gradient} to a condition named "quasi-convexity", which can be written as a Jensen-type inequality
\begin{equation} \label{Introduction:eq:quasi_convexity}
\int_\Omega f(z+D\phi(x))dx \geq |\Omega|f(z) \quad \forall \phi\in \D(\Omega).
\end{equation}
The original result by Morrey deals with weak* convergence in $W^{1,\infty}(\Omega,\R^m)$, and it was subsequently extended to the case $W^{1,p}(\Omega,\R^m), 1\leq p<\infty$ and weak convergence in \cite{acerbi1984semicontinuity}, for positive integrands. As for signed integrands, the same result was proven in \cite{ball1990lower} and it is one of the first examples where Young Measures are employed for proving lower semi-continuity in the space of gradients. To be more specific, \eqrefp{Introduction:eq:quasi_convexity} can be rephrased as a Jensen-type inequality for measures of the form
\begin{equation} \label{pullback_gradients}
\{ \nu_x: \nu_x = D\phi(x)_\# d\L^n\mres \Omega, \phi\in C_c^\infty(\Omega, \R^m) \},
\end{equation}
where $\nu_x$ acts on $f$ by
\begin{equation}
\int_\Omega f(z+D\phi(x))dx = \int_\Omega \int_{\R^{m\times n}} f(z+w) d\nu_x(w) dx \equiv \int_\Omega \langle \nu_x,f\rangle dx.
\end{equation}
Therefore, lower semi-continuity of \eqrefp{Introduction:eq:functional_gradient} becomes a functional analytic inequality of the form
\begin{align}
\int_\Omega \langle \nu_x,f\rangle dx \geq \int_\Omega f(Du(x))dx \quad \text{and} \quad Du(x) = \int_{\R^{m\times n}} zd\nu_x.
\end{align}
where these new measure-valued functions $x\mapsto \nu_x$ constitute a special class of Young Measures called Gradient Young Measures (abbreviated by GYM). This class can be seen as the closure of the set \eqrefp{pullback_gradients} in the weak* topology of measures over the graph of $f$. In particular, \eqrefp{Introduction:eq:functional_gradient} holds for all $\nu_x\in GYM$. The opposite is also true and was proven for the first time in \cite{kinderlehrer1991characterizations,kinderlehrer1994gradient}, i.e. every measure valued function $x\mapsto \nu_x$ for which a Jensen's type inequality holds against quasi-convex functions of suitable growth is limit of a sequence of gradients.

The aforementioned results hold in the setting of weak convergence in $W^{1,p}, 1\leq p<\infty$ and weak* convergence in $W^{1,\infty}$. This is a natural condition to assume when $p>1$, but not when $p=1$, as the Lebesgue space $L^1(\Omega,\L^n)$ is not reflexive. In particular, a bounded sequence in $L^1(\Omega,\L^n)$ can concentrate and give rise to measures that are singular with respect to the Lebesgue measure. In terms of gradients, the closure of $W^{1,1}(\Omega)$ so that its unit ball is weak* compact is the set of functions of bounded variations $BV(\Omega)$, precisely the set of functions whose derivatives are measures. This concentration phenomenon is exclusive of the case $p=1$, and so applies to integrands that have linear growth at infinity. It turns out, as proven in \cite{ambrosio1992relaxation}, that when $f$ has linear growth and it's quasi-convex, the integral functional $u\mapsto \int_\Omega f(\nabla u)dx,f\geq 0$ is still lower semi-continuous in $BV(\Omega,\R^m)$ with respect to the weak* topology, but there is a decay of mass when $u$ concentrates. Letting $f$ be so that
\begin{align}
f^\infty(z) = \lim_{t\to \infty,z_n\to z} \frac{f(tz_n)}{t}
\end{align}
exists for all $z_n\to z, t\to \infty$, the lower semi-continuous envelope of \eqrefp{Introduction:eq:functional_gradient} in the space $BV(\Omega,\R^m)$, with respect to sequential weak* convergence, is
\begin{equation}
u\mapsto \int_\Omega f(\nabla u(x))dx + f^\infty \left( \frac{D^s u}{|D^s u|}(x) \right) d|D^s u|(x)
\end{equation}
(again $f$ is assumed to be positive). In this case, a Young Measure formulation of the Jensen's inequality \eqrefp{Introduction:eq:quasi_convexity} has to take into account the singular part of $Du$. In spirit of the previous results, one is tempted to prove a duality characterisations of Young Measures with concentrations and quasi-convex functions. Differently from the case without concentration, here we assumed $f^\infty$ to exist. However, as shown in \cite{muller1992quasiconvex}, quasi-convex functions can oscillate at infinity, as in $f^\infty(z)$ needs not to exist at all $z\in \R^m$. This suggests that to obtain a Jensen-type inequality and characterisation result for gradient Young Measure in the case $p=1$, it is necessary to specify a compactification at infinity.

The characterisation for gradient Young Measures when $p=1$ have been already obtained on the so-called "sphere compactification" - functions for which $f^\infty(z)$ exists for all $z$ - see \cite{kristensen2010characterization,kristensen2010relaxation}.  After showing that the whole class of quasi-convex functions of linear growth is far too big to be included within any separable compactification, we reprove a characterisation result for gradient Young Measures on separable compactifications of quasi-convex functions. This clearly restrict the number of quasi-convex functions to be considered at once. However, it is also inevitable because a compactification containing all quasi-convex functions would be so big so that its topology would fail to be metrisable and separable. This is really unsettling as the lack of separability prevents sequential compactness and "good" representations of Young Measures. \\